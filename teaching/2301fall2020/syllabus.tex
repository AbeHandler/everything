 \documentclass[10pt]{memoir}

% based on kieran healy's memoir modifications
\usepackage{mako-mem}
\chapterstyle{article-2}
\pagestyle{mako-mem}

\usepackage{ucs}
\usepackage[utf8x]{inputenc}

%\usepackage{kpfonts}
%\usepackage[bitstream-charter]{mathdesign}
\usepackage{fbb}
\usepackage[T1]{fontenc}
%\usepackage{textcomp}

%\renewcommand{\rmdefault}{ugm}
%\renewcommand{\sfdefault}{phv}

% Packages for making a landscape table (?)
\usepackage[table,usenames,dvipsnames]{xcolor}
\usepackage{multirow, makecell}
\usepackage{pdflscape}
\usepackage{afterpage}

\usepackage[letterpaper,left=1.125in,right=1.125in,top=1.25in,bottom=1.25in]{geometry}

\usepackage{datenumber}
\setdate{2020}{8}{24}
\def\datedate{\datedayname,\space\datemonthname~\thedateday}

\newcommand{\adddays}[1]{%
    \addtocounter{datenumber}{#1}%
    \setdatebynumber{\thedatenumber}%
}

% Common packages
\usepackage{graphicx}
\usepackage{enumerate}

% Setup list environments
\usepackage{enumitem}
\setlist[description]{
  topsep=0pt,
  before=\vspace{0pt},
  after=\vspace{0pt},
  itemsep=2pt,
  labelsep=0pt
}

\setlist[itemize]{
    noitemsep, 
    leftmargin=1em,
    topsep=0pt}

% Set paragraph indents and spacing
\setlength{\parindent}{0pt}
\setlength{\parskip}{.5\baselineskip}
%\usepackage[document]{ragged2e}

% adjust section title formatting
\usepackage{titlesec}
\titlespacing\section{0pt}{8pt plus 2pt minus 2pt}{-6pt}
\titlespacing\subsection{0pt}{8pt plus 2pt minus 2pt}{-6pt}
\titlespacing\subsubsection{0pt}{8pt plus 2pt minus 2pt}{10pt}

% allows full, in-line citations
\usepackage{bibentry} 

% add bibliographic stuff 
\usepackage[round, numbers]{natbib} \def\citepos#1{\citeauthor{#1}'s (\citeyear{#1})} \def\citespos#1{\citeauthor{#1}' (\citeyear{#1})}
\renewcommand{\bibnumfmt}[1]{}

% define colors from http://www.colorado.edu/brand/visual-identity/typography-color
%\usepackage[usenames,dvipsnames]{color}
\definecolor{CUGold}{RGB}{207,184,124}
\definecolor{CUDarkGray}{RGB}{86,90,92}
\definecolor{CULightGray}{RGB}{162,164,163}

% customize URLs
\usepackage[hyphens]{url}
%\usepackage{breakurl} 
\usepackage[breaklinks, bookmarks, bookmarksopen]{hyperref}
\hypersetup{
    colorlinks=true,
    linkcolor=Blue,
    citecolor=Black,
    filecolor=Blue,
    urlcolor=Blue,
    unicode=true,
    breaklinks=true}

% create a "reading list" environment to format the items
\newenvironment{readinglist}{
\begin{list}{}{\leftmargin=0pt \itemindent=0em}
  \setlength{\itemsep}{8pt}
  \setlength{\parskip}{0em}
  \setlength{\parsep}{1em}
  \setlength{\parindent}{8em}}
{\end{list}}

% Course/Instructor metadata -- alter as needed
\def\mycoursename{Quantitative Reasoning for Information Science}
\def\mycourselisting{INFO 2301}
\def\myclassroom{Remote}
\def\myzoomurl{TODO}
\def\mycanvasurl{https://canvas.colorado.edu/courses/62559}
\def\mymeetingdays{Monday, Wednesday, Friday}
\def\mymeetingtimes{11:30AM--12:20PM}
\def\mydate{Fall 2020}

\def\instructorAfirstname{Abram}
\def\instructorAlastname{Handler}
\def\instructorAfullname{\instructorAfirstname~\instructorAlastname, M.S.}
\def\instructorAtitle{Instructor, Information Science}
\def\instructorAoffice{INFO TBD}
\def\instructorAemail{abram.handler@colorado.edu}
\def\instructorAwebsite{https://www.abehandler.com/}
\def\instructorAofficehours{TBD}



\begin{document}

\nobibliography*

%\baselineskip 14.2pt

\title{
    \textbf{\huge{\mycoursename}}\\
    \vspace{5pt} \normalsize{\mycourselisting}; \mydate
}

\author{
    \mymeetingdays; \mymeetingtimes\\
    \myclassroom\\
}

\date{
    \normalsize{
        \href{\instructorAwebsite}{\textbf{\instructorAfullname}}\\
        \instructorAtitle\\
        E-mail: \href{mailto:\instructorAemail}{\instructorAemail}\\
        Office: \instructorAoffice\\
        Office hours: \instructorAofficehours\\\vspace{1em}
        
    }
}

\maketitle

%%%%%%%%%%%%%%%%%%%%%%
%% Acknowledgements %%
%%%%%%%%%%%%%%%%%%%%%%
% This syllabus template was made in LaTeX by Brian Keegan and is distributed as Free Software under the GNU GPL v3. It was built using style templates created by Aaron Shaw, Benjamin Mako Hill, and Kieran Healy.

TODO: quizzes, expectations about that, modules

\section{\textbf{Course Description}}

Introduces methods for quantifying and analyzing different types of data, covering foundational
concepts in discrete mathematics, probability, and predictive modeling, along with complementary computational skills to apply these concepts to real problems. Covers counting and combinatorics, logic, set theory, introductory probability, common probability distributions, regression,
and model validation. Requires demonstrated proficiency with introductory computer programming.

The following topics will be covered

\begin{itemize}
\item Data types and corresponding data structures in Python
\item Operations on data structures (sets and vectors)
\item Counting and combinatorics (combinations and permutations)
\item Probability concepts (independence, marginalization, expected value)
\item Probability distributions (discrete and continuous)
\item Conditional probability (language modeling and music generation as applications)
\item Probability in Python (distributions, pseudorandom sampling)
\end{itemize}


\section{\textbf{Cumulative Instruction}}

This course covers foundational mathematical concepts that you will use again and again in working with data. In order to help you master the material, the course will be taught \textit{cumulatively}. That means that quizzes, lectures and homeworks will revisit material from earlier in the semester, over and over again. 

Because the course is \textit{cumulative}, as you study for quizzes and tests, you'll want to remind yourself of everything we've covered in the course so far, as you will see questions from material across the semester. Additionally, if you don't understand a concept, it will be very important to come to office hours to get help. You will very likely see the concept on future assignments and assessments, not to mention in your future life as an information scientist!

\section{Textbook and software }

\begin{itemize}
\item \textbf{Textbook}: \href{https://www.abehandler.com/resources/openintro-statistics.pdf}{OpenIntro} Statistics (free)
\item \textbf{Software}: Assignments will use \href{https://www.python.org/}{Python 3} in \href{https://jupyter.org/}{Jupyter notebooks}
\end{itemize}


\section{Learning Goals}

By the end of this semester you will be able to:
\begin{itemize}
\item Understand the different types of mathematical objects, and how to implement those objects in code
\item Learn how to reason under uncertainty
\item Solve problems involving probabilistic data
\item Randomly generate data 
\item Run simulations in code
\end{itemize}

\section{Course Design}
Class will meet three times per week (\mymeetingdays)\space from \mymeetingtimes\space on Zoom. \textit{Attendance is required during class time.} 

This course will largely follow a \href{https://en.wikipedia.org/wiki/Flipped_classroom}{``flipped classroom''}  model. You will watch pre-recorded lectures before class. During class, we will discuss and practice what you have learned from pre-recorded lectures. 

You will be expected to have watched the lectures before starting class. During class, there will be regular quizzes to assess what you have learned from pre-recorded videos. 

In order to fully participate remotely, students will need access to a personal computer with a web camera, a reliable high-speed internet connection, and a minimally disruptive background environment. Please email the instructor immediately if you expect any issues with remote participation.

\section{Prerequisites}

Requires prerequisite course of INFO 1201 or CSCI 1200 or CSCI 1300 or LING 1200 or ATLS 1300 (all minimum grade C-). If you have questions, please email the instructor.

\section{Late work policy}
The instructor understands that students are busy with many obligations. Therefore, across the whole semester, you will be allotted a total of five free late days to turn in assignments. If you turn in your assignment within 24 hours after the deadline you will be deducted 1 late day, if you turn in in your assignment within 48 hours after the deadline you will be deducted 2 late days, and so on. 

If you have used five or fewer late days so far during the semester (including the most recent assignment) you will not be penalized for late work. However, after you have used up your late days, late homework will not count for credit except in special circumstances.

\section{Grading}

Final grades are calculated according to the following distribution:
\begin{itemize}
\item 5\% Attendance
\item 10\% Quizzes (in class and/or online)
\item 40\% Homework
\item 20\% Midterm Exam
\item 25\% Final Exam
\end{itemize}

Letter grades will follow a typical scoring distribution (A if >= 93\%, A- if >= 90\%, B+ if >=
87\%, B if >= 83\%, B- if >= 80\%, C+ if >= 77\%, and so on). Do not expect most grades to be
curved, though exam grades may be curved if needed.

% \subsection{Requirements}
% Students' regular and sustained participation in all class activities as well as punctual and thorough completion of assignments are essential. If you need to be excused from attending a class session or need an extension to an assignment, please \href{mailto:brian.keegan@colorado.edu}{email instructor} at least 24 hours in advance.

% \clearpage
\subsection{Course Website and Materials}
There is no textbook required for class, but there will be required readings, tutorials, and other material, which will be made available through Canvas:
\vspace{-8pt}
    \begin{center}
    \Large{\textbf{Canvas}: \href{\mycanvasurl}{\mycanvasurl}}\\
    \Large{\textbf{Zoom}: \href{\myzoomurl}{\myzoomurl}}
    \end{center}
\vspace{-8pt}
Once the semester begins, this PDF version of the syllabus will be revised infrequently and any revised requirements will be posted as announcements and updated course schedule to Canvas. The instructors reserve the right to make changes to the course's schedule, evaluation criteria, policies, \textit{etc.} through announcements in class and on Canvas, so please check Canvas regularly. If you have questions, please email \href{mailto:\instructorAemail}{Dr. \instructorAlastname}.

% The class will make extensive use of the \href{https://www.medium.com}{Medium} blogging platform. Instruction on how to create accounts, read, write, and post to the \href{https://medium.com/information-expositions}{class publication} will be covered in the first week of class. There is extensive documentation in the \href{https://help.medium.com/hc/en-us}{Medium Help Center} as well as multiple tutorials. Students will write their Module Assignments on Medium and submit links via Canvas with the expectation that their writing will be read by the general public. If students are unable or do not want to use the Medium platform, they should \href{mailto:brian.keegan@colorado.edu}{email the instructor} \textbf{immediately} to work out an alternative arrangement before Wednesday, September 4.

\subsection{Computing Requirements}
Students will need to use statistical computing software as well as teleconferencing software to participate in class. \href{http://jupyter.org/}{Jupyter notebooks} written in Python 3 will be used for all in-class examples and assignments. The \href{https://www.continuum.io/why-anaconda}{Anaconda distribution} of Python 3.5 (or above) is \textit{strongly} recommended to provide all of these programs and other libraries. Lectures will include exercises and presentations with the expectation that students participate with their own computers. If students do not have access to a computer to use for computing or Zoom, they should immediately email the instructor. Students are welcome to use an alternative \textit{programmatic} (\textit{not} Excel or Tableau) data analysis environment like R, Matlab, Julia, \textit{etc.}, but instructional support will only be provided for Anaconda and Python. Students who require technical assistance should email the instructors with the code and data they are working with, a summary of their debugging efforts to date, and attend an instructor's office hours.


\section{Course Policies}

% \subsection{Professional Expectations}
% In addition to the expectations outlined in the \href{https://www.colorado.edu/creed/}{Colorado Creed}, we expect students to conduct themselves in a professional manner in their interactions

\subsection{In-Class Confidentiality}
The success of this class depends on students feeling comfortable sharing questions, ideas, concerns, and confusions about assignments, work-in-progress, and their personal experiences. Students may read, comment, and run on classmates' writing, code, and other class-related content for the sole purpose of use within this class. However, students may not use, run, copy, perform, display, distribute, modify, translate, or create derivative works of another student's work outside of this class without that student's expressed written consent or formal license. Furthermore, students may not create any audio, video, or other records during class time without the instructor's permission nor may students publicly share comments made in class attributable to another person's identity without that person's permission.

% \subsection{Critical Response Process}
% The class will make regular use of Liz Lerman's ``\href{https://lizlerman.com/critical-response-process/}{Critical Response Process}'' for the Weekly Presentations and the Final Project. The details of this process will be covered in more detail in Week 1, but students will rotate through the roles of ``artist'' and ``responder'' regularly. Students are expected to participate in good faith when sharing statements of meaning, questions, neutral questions, and permissioned opinions.

\subsection{Instructor Interaction}
The instructor will check e-mail between 8:00 and 18:00 on non-holiday business days and try to respond to emails within 24 hours. They welcome online or offline interactions outside of class, however these are not appropriate spaces for discussing class matters. E-mailing the instructor or coming to (remote) office hours are the best ways to get help and feedback outside of lecture.

\subsection{Accommodations for Disabilities}
We are committed to providing everyone the support and services needed to participate in this course. If you qualify for accommodations because of a disability, please submit your accommodation letter from Disability Services to the instructor in a timely manner so that your needs can be addressed. Disability Services determines accommodations based on documented disabilities in the academic environment. Information on requesting accommodations is located on the \href{Disability Services website}{www.colorado.edu/disabilityservices/students}. Contact Disability Services at 303-492-8671 or \href{mailto:dsinfo@colorado.edu}{dsinfo@colorado.edu} for further assistance. If you have a temporary medical condition or injury, see Temporary Medical Conditions under the Students tab on the Disability Services website and discuss your needs with the instructors.

\subsection{Religious Observance}
Campus policy regarding \href{http://www.colorado.edu/policies/observance-religious-holidays-and-absences-classes-andor-exams}{religious observances} requires that faculty make every effort to deal reasonably and fairly with all students who, because of religious obligations, have conflicts with scheduled exams, assignments or required assignments/attendance. If this applies to you, please e-mail \href{\instructorAemail}{Dr. Handler} as soon as possible to make the appropriate accommodations.

\subsection{Classroom Behavior}
Students and instructors each have responsibility for maintaining an appropriate learning environment. Those who fail to adhere to such behavioral standards may be subject to discipline. Professional courtesy and sensitivity are especially important with respect to individuals and topics dealing with differences of race, color, culture, religion, creed, politics, veteran’s status, sexual orientation, gender, gender identity and gender expression, age, ability, and nationality. Class rosters are provided to the instructor with the student's legal name. The instructor will honor your request to address you by an alternate name or gender pronoun. Please advise the instructors of this preference early in the semester so that we may make appropriate changes. For more information, see the policies on \href{http://www.colorado.edu/policies/student-classroom-and-course-related-behavior}{class behavior} and the \href{http://www.colorado.edu/osc/#student_code}{student code}.

\subsection{Harassment and Discrimination}
The University of Colorado Boulder (CU Boulder) is committed to maintaining a positive learning, working, and living environment. CU Boulder will not tolerate acts of sexual misconduct, discrimination, harassment or related retaliation against or by any employee or student. CU's \href{http://www.colorado.edu/policies/discrimination-and-harassment-policy-and-procedures}{Sexual Misconduct Policy} prohibits sexual assault, sexual exploitation, sexual harassment, intimate partner abuse (dating or domestic violence), stalking or related retaliation. CU Boulder's \href{http://www.colorado.edu/policies/discrimination-and-harassment-policy-and-procedures}{Discrimination and Harassment Policy} prohibits discrimination, harassment or related retaliation based on race, color, national origin, sex, pregnancy, age, disability, creed, religion, sexual orientation, gender identity, gender expression, veteran status, political affiliation or political philosophy. Individuals who believe they have been subject to misconduct under either policy should contact the Office of Institutional Equity and Compliance (OIEC) at 303-492-2127. Information about the OIEC, the above referenced policies, and the campus resources available to assist individuals regarding sexual misconduct, discrimination, harassment or related retaliation can be found at the \href{http://www.colorado.edu/institutionalequity/}{OIEC website}.

\subsection{Honor Code}
All students enrolled in a University of Colorado Boulder course are responsible for knowing and adhering to the \href{http://www.colorado.edu/policies/academic-integrity-policy}{academic integrity policy} of the institution. Violations of the policy may include: plagiarism, cheating, fabrication, lying, bribery, threat, unauthorized access to academic materials, clicker fraud, resubmission, and aiding academic dishonesty. All incidents of academic misconduct will be reported to the Honor Code Council (\href{mailto:honor@colorado.edu}{honor@colorado.edu}; 303-735-2273). Students who are found responsible for violating the academic integrity policy will be subject to nonacademic sanctions from the Honor Code Council as well as academic sanctions from the faculty member. Additional information can be found at \href{http://honorcode.colorado.edu}{honorcode.colorado.edu}. 

\subsection{Illness}
Should a student contract any illness that requires mandatory sequestration, intensive medical treatment, or extended convalescence and disrupts their ability to participate in class and complete assignments, the instructors will try to accommodate their condition without penalty with extensions and incompletes. This also applies if the student has a family member whose diagnosis, treatment, and recovery will affect their ability to participate. \textit{Please do not ghost us}: students should notify the instructors as soon as possible of events that will impact their engagement with the class so that we can triage and develop an accommodation plan rather than scrambling at the end of the semester.

\section{Acknowledgements}
% The design and format of this course borrows from other courses.
% \begin{itemize}[itemsep=1em]
%     \item Bergstrom, Carl \& West, Jevin (2017). \href{http://callingbullshit.org/index.html}{\textit{Calling Bullshit}}. University of Washington.
% \end{itemize}

This syllabus was typeset in \LaTeX~using \href{http://www.sharelatex.com}{Overleaf} with the \href{http://www.tug.dk/FontCatalogue/fbb/}{fbb/Bembo} font and is derived from the \texttt{memoir} styles adapted by \href{https://github.com/kjhealy/latex-custom-kjh}{Kieran Healy} and \href{http://projects.mako.cc/source/?p=latex_mako;a=summary}{Benjamin `Mako' Hill}.

%%%%%%%%%%%%%%%%%%%%%%
%%%%%%%%%%%%%%%%%%%%%%
%%% COURSE OUTLINE %%%
%%%%%%%%%%%%%%%%%%%%%%
%%%%%%%%%%%%%%%%%%%%%%
\clearpage

%\section{\textbf{Course Outline}}

%The schedule will evolve throughout the semester, so please consult the schedule online at Canvas for the most up-to-date information.

% \section{Week 1 -- Introductions}
%  \textcolor{CUGold}{\textbf{Monday, August 24; Wednesday, August 26; Friday, August 28}}\\
% \textcolor{CUGold}{\textbf{%
%   \adddays{0}\datedate;
%   \adddays{2}\datedate;
%   \adddays{2}\datedate}}\\
% Course overview; configuring environment; building a data scientist mindset.

% \section{Week 2 -- Tabular data: Fundamentals} % Brian
% \textcolor{CUGold}{\textbf{%
%     \adddays{3}\datedate;
%     \adddays{2}\datedate;
%     \adddays{2}\datedate}}\\
% Reviewing tabular data structures, file I/O; long \textit{vs}. wide data; tabular data with \href{https://pandas.pydata.org/}{\texttt{pandas}}.

% \section{Week 3 -- Tabular data: Applications} % Brian
% \textcolor{CUGold}{\textbf{%
%     \adddays{5}\datedate;
%     \adddays{2}\datedate}}\\
%     \textbf{(No class on Monday, September 7 in observance of Labor Day)}\\
% Cleaning and analyzing data about populations from \href{https://data.census.gov/}{U.S. Census}, \href{https://data.un.org/}{United Nations}, \href{https://www.who.int/gho/database/en/}{WHO}, \textit{etc}.

% \section{Week 4 -- Relational data: Fundamentals} % Abe
% \textcolor{CUGold}{\textbf{%
%     \adddays{3}\datedate;
%    \adddays{2}\datedate;
%     \adddays{2}\datedate}}\\
% Understanding related data across multiple tables; types of joins; databases with \href{https://www.sqlite.org/index.html}{\texttt{sqlite}}.

% \input{M2-relational}

% \section{Week 5 -- Relational data: Applications} % Abe
% \textcolor{CUGold}{\textbf{%
%     \adddays{3}\datedate;
%     \adddays{2}\datedate;
%     \adddays{2}\datedate}}\\
% Administering a simple database; analyzing historical baseball data from \href{https://www.retrosheet.org/}{Retrosheet} and \href{http://www.seanlahman.com/baseball-archive/statistics/}{Lahman Database}.

% \section{Week 6 -- Temporal data: Fundamentals}
% \textcolor{CUGold}{\textbf{%
%     \adddays{3}\datedate;
 %    \adddays{2}\datedate;
 %    \adddays{2}\datedate}}\\
% Understanding time series data; auto-correlation, change-points, anomalies; forecasting using \href{https://facebook.github.io/prophet/}{\texttt{prophet}}.


% \clearpage
\vspace{2em}

\begin{table}[h]
\centering
\begin{tabular}{cccl}
    \toprule[.15em]
    \textbf{Module} & \textbf{Week} & \textbf{Dates} & \textbf{Topics} \\
    \cmidrule[.1em](lr){1-4}
    
    & 1 & Aug 24 -- Aug 28 & Introductions, command line \textit{vs}. Jupyter \\
    \cmidrule[.1em](lr){1-4}
        
    \multirow{2}{*}[0pt]{\textit{Tabular}} % Brian
        & 2 & Aug 31 -- Sep 4 & \textbf{Fundamentals}: single tables, \texttt{pandas} \\ 
        & 3 & Sep 9 -- Sep 11 & \textbf{Applications}: population data \\ \cmidrule[.1em](lr){1-4}
    
    \multirow{2}{*}[0pt]{\textit{Relational}} % Abe % sql lite. Brian has notebooks
        & 4 & Sep 14 -- Sep 18 & \textbf{Fundamentals}: multiple tables, \texttt{sqlite} \\ 
        & 5 & Sep 21 -- Sep 25 & \textbf{Applications}: sports data \\ 
        \cmidrule[.1em](lr){1-4}
        
    \multirow{2}{*}[0pt]{\textit{Temporal}} % Brian
        & 6 & Sep 28 -- Oct 2 & \textbf{Fundamentals}: time series, \texttt{prophet} \\
        & 7 & Oct 5 -- Oct 9 & \textbf{Applications}: economic data  \\ 
        \cmidrule[.1em](lr){1-4}
        
    \multirow{2}{*}[0pt]{\textit{Spatial}} % Brian
        & 8 & Oct 12 -- Oct 16 & \textbf{Fundamentals}: mapping, \texttt{geopandas} \\
        & 9 & Oct 19 -- Oct 23 & \textbf{Applications}: political data \\ \cmidrule[.1em](lr){1-4}
    
    \multirow{2}{*}[0pt]{\textit{Dyadic}} % Brian
        & 10 & Oct 26 -- Oct 30 & \textbf{Fundamentals}: network analysis, \texttt{networkx} \\
        & 11 & Nov 2 -- Nov 6 & \textbf{Applications}: social network data \\ \cmidrule[.1em](lr){1-4}
        
    \multirow{2}{*}[0pt]{\textit{Structured}} % Abe
        & 12 & Nov 9 -- Nov 13 & \textbf{Fundamentals}: JSON \& XML, \texttt{BeautifulSoup} \\
        & 13 & Nov 16 -- Nov 20 & \textbf{Applications}: API data \\ 
        \cmidrule[.1em](lr){1-4}
    
    & 14 & Nov 23 -- Nov 27 & Not Fall Break - Catch up \\ \cmidrule[.1em](lr){1-4}
    
    \multirow{2}{*}[0pt]{\textit{Unstructured}} % Abe //// grep 
        & 15 & Nov 30 -- Dec 4 & \textbf{Fundamentals}: text processing, \texttt{nltk} \\ 
        & 16 & Dec 7 -- Dec 11 & \textbf{Applications}: text data \\
        
    \bottomrule[.15em]
\end{tabular}\\
\end{table}

\renewcommand{\bibsection}{\section{\huge \bibname}\prebibhook}
\baselineskip 14.2pt
\nobibliography{refs}
\bibliographystyle{apalike}

\end{document}
