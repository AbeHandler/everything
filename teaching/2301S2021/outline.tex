%%%%%%%%%%%%%%%%%%%%%%%%%%%%%%%%%%%%%%%%%%%%%%%%%%%%%%%%%%%%%%%%%%%%%%%%%%%%%%%%%%%%%%%%%%%%%%%%%%%%%%%%%%%%%
% % % % % % % % % % % % % % % % % % % % % % % % % % % % % % % % % % % % % % % % % % % % % % % % % % % % % % % 
% = = = = = = = = = = = = = = = = = = = = = = = = = = = = = = = = = = = = = = = = = = = = = = = = = = = = = =
%
% This is where the packages are declared

\documentclass[]{article}

\usepackage[english]{babel}
\usepackage[utf8]{inputenc}
\usepackage{amsmath}
\usepackage{amsthm}
\usepackage{graphicx}
\usepackage{dirtytalk}
\theoremstyle{definition}
\newtheorem{exmp}{Example}[section]

\begin{document}


\title{INFO 2301: Quantitative Reasoning for Information Science}
\author{Abe Handler \\ Department of Information Science \\ University of Colorado, Boulder}
\date{\today}

\maketitle

\begin{abstract}
An outline of what we've covered so far. You should be comfortable with everything in this document. 
\end{abstract}

\section{Sets}

A \textbf{set} is an unordered collection of items, without duplicates. 

People follow a number of conventions when talking about sets.

\begin{itemize}
    \item The items in a set are often called ``elements''
    \item We use capital letters to name sets
    \item We show the elements in a set using curly brackets
    \item A set can include any kind of element (e.g.\ strings, or kinds of apple)
\end{itemize}

\begin{exmp}
We can write the integers between 0 and 3 (including 0 and 3) as $A$ = $\{0,1,2,3\}$
\end{exmp}

\begin{exmp}
Because sets do not have order, if $ B= \{0,1\}$ and $C=\{1,0\}$, then $C=B$.
\end{exmp}

\begin{exmp}
A set of strings, $ P= \{ \text{\say{Denver},\say{Boulder},\say{Broomfield}} \}$
\end{exmp}

\begin{itemize}
    \item If $a$ is an element in $X$ then, we use the notation $a \in X$ 
    \item If $a \in X$ then we say that ``$a$ is in $X$''
    \item We write this: \verb|$a \in X$|
    \item If $a \notin X$ then we say that ``$a$ is not in $X$''
    \item We write this: \verb|$a \notin X$|
\end{itemize}

\begin{exmp}
Let $ C= \{1, 2\}$. Then $1 \in C$ and $6 \notin C$
\end{exmp}


\subsection{Subsets}

If all elements of $A$ are elements of $B$ then we say that $A$ is a subset of $B$.  We use the notation $ A \subset B$ (written \verb|$ A  \subset B $|) to indicate that $A$ is a subset of $B$.

\begin{exmp}\label{e:subset}
Let $ A= \{2, 1\}$ and $B = \{1,2,5\}$. Then $A \subset B$
\end{exmp}

In this class, when we use $ A \subset B$ we will assume that some elements of $B$ are not elements of $A$. So $A$ is a ``proper subset'' of $B$. For instance, in Example \ref{e:subset}, $A \subset B$ because $5 \in B$ and $5 \notin A$. If we want to allow for the possibility that $A=B$, we can use the notation $A \subseteq B$ (written \verb|$ A \subseteq B $|). 

\begin{exmp}
Let $ A= \{1, 2,3 \}$ and $B = \{1,2,3\}$. Then $A \subseteq B$ but not $A \subset B$
\end{exmp}

\begin{exmp}
Let $ A= \{1, 3 \}$ and $B = \{1,2,3\}$. Then $A \subseteq B$ and $A \subset B$
\end{exmp}


\textit{If you spot any errors in this document, or if anything is needlessly confusing, please send me an email. I will assign extra credit.}

\end{document}