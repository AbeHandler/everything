%%%%%%%%%%%%%%%%%%%%%%%%%%%%%%%%%%%%%%%%%%%%%%%%%%%%%%%%%%%%%%%%%%%%%%%%%%%%%%%%%%%%%%%%%%%%%%%%%%%%%%%%%%%%%
% % % % % % % % % % % % % % % % % % % % % % % % % % % % % % % % % % % % % % % % % % % % % % % % % % % % % % % 
% = = = = = = = = = = = = = = = = = = = = = = = = = = = = = = = = = = = = = = = = = = = = = = = = = = = = = =
%
% This is where the packages are declared

\documentclass[]{article}

\usepackage[english]{babel}
\usepackage[utf8]{inputenc}
\usepackage{amsmath}
\usepackage{amsthm}
\usepackage{graphicx}
\theoremstyle{definition}
\newtheorem{exmp}{Example}[section]

\begin{document}


\title{INFO 2301: Quantitative Reasoning for Information Science}
\author{Abe Handler \\ Department of Information Science \\ University of Colorado, Boulder}
\date{\today}

\maketitle

\begin{abstract}
An outline of what we've covered so far. You should be comfortable with everything in this document. 
\end{abstract}

\section{Sets}

\begin{itemize}
    \item A \textbf{set} is an unordered collection of items, without duplicates
    \item The items in a set are often called ``elements''
    \item We use capital letters to name sets
    \item We show the elements in a set using curly brackets
\end{itemize}

\begin{exmp}
Using postal code abbreviations, we can write the set of states which border Colorado as  $A$ = $\{UT, AZ, NM, OK, KS, NE, WY\}$
\end{exmp}

\begin{exmp}
Because sets do not have order, if $ B= \{UT, AZ\}$ and $C=\{AZ, UT\}$, then $C=B$.
\end{exmp}

\begin{itemize}
    \item If $a$ is an element in $X$ then, we use the notation $a \in X$ 
    \item If $a \in X$ then we say that ``$a$ is in $X$''
    \item We write this: \verb|$a \in X$|
    \item If $a \notin X$ then we say that ``$a$ is not in $X$''
    \item We write this: \verb|$a \notin X$|
\end{itemize}

\begin{exmp}
Let $ C= \{UT, AZ\}$. Then $UT \in C$ and $FL \notin C$
\end{exmp}

\end{document}